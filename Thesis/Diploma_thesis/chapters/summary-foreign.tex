%----------------------------------------------------------------------------
\chapter*{\summary}\addcontentsline{toc}{chapter}{\summary}
%----------------------------------------------------------------------------

\selectforeignlanguage % angol (magyar) nyelvi beállítások

I designed and implemented a six-degree-of-freedom telemanipulator. The telemanipulator incorporates gravity and mass compensatory kinematic solutions.

I also designed an end-effector to attach to the device, and throughout the entire design process, I kept in mind that any tool could be connected to it.

My main goal was to use angle signals collected on the telemanipulator for robot control with the help of an STM32 microcontroller. I successfully achieved this set goal. I created a telemanipulator and, utilizing the signal processing system, applied kinematic calculations to provide control commands to a simulated environment and a real collaborative robot. In both cases, the control remained stable, allowing me to continuously operate the robot and observe the characteristics of my current setup.

During control, some technical issues arose, as the robot exhibited vibrations around a stable point. Towards the end of my thesis, my objective was to find a solution to this oscillation issue.

I acquired during my studies enabled me to gather information from the initial physical world using sensors on the telemanipulator about the operator's movement intentions. I transformed this information into control commands in a closed software environment, and by transmitting this, the robot implemented it in the physical world.

\vspace{0.5cm}
\paragraph{Keywords} \emph{\keywords}  % A kulcsszavak a fő tex fájlban vannak definiálva


\selectthesislanguage % térjünk vissza magyar (angol) nyelvre
