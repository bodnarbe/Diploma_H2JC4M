%----------------------------------------------------------------------------
\chapter*{\eloszo}\addcontentsline{toc}{chapter}{\eloszo}
%----------------------------------------------------------------------------

Az általam választott témának az a meghatározó alapja, hogy a fizikai világból információt
gyűjtsünk és azzal valamilyen adatfeldolgozó műveletet végre hajtsunk, végigkísérte
az egyetemi féléveimet. Mechatronikai alapképzése során elkészített telemanipulátort követően azonnal tudtam, hogy ugyanebben a témában szeretném a diplomadolgozatomat is írni. A témában programozási ismeretek, 3D tervezés és a mikrovezérlő programozás is része, ami eredményeképpen szerteágazó ismeretre van szükség. A témaválasztásom célja az volt, hogy a korában elkészített telemanipulátoromat továbbfejlesszem. A szakdolgozatomban elkészített eszközt követően azonnal voltak ötleteim, hogy milyen irányba is szeretném az eszközt fejleszteni. Ebben az esetben már több időm volt, így több féléven keresztül kísérleteztem azzal, hogy mi is lenne a tökéletes megoldás. A végeredmény nem állítom, hogy minden kérdést megválaszolt, amit a szakdolgozatom végén feltettem, de egy elkészült és érdekes újításokkal felszerelt eszköz lett. A következőkben az ehhez elvégzett munkát mutatom be.

\begin{center}
    $\thicksim \; \thicksim \; \thicksim$
\end{center}


\subsubsection*{Köszönetnyilvánítás}
\emph{Szeretnék köszönetet mondani témavezetőmnek, Dr. Budai Csabának és konzulensemnek Dudás Dávidnak a több féléves munkájukért. Szeretném megköszönni Dr. Bodnár Tibor, Bodnárné Dr. Gyarmathy Dórának és Dr. Gyarmathy Miklósnénak a támogatást és a lektorálást. Szeretnék köszönetet mondani Dr. Kiss Rita tanészékvezető asszonynak és teljes tanészéknek a támogatásért. Szeretném megköszönni az Orvostechnika Szakosztály tagjainak a projektben vállalt kisebb feladatokban nyújtott segítséget, illetve a lehetőséget, hogy ezzel a projekttel hozzájárulhattam a Szakosztály életéhez.} 


\vspace{0.5cm}

\begin{flushleft}
{Budapest, \today}
\end{flushleft}

\begin{flushright}
\emph{\authorName}
\end{flushright}

\vfill
