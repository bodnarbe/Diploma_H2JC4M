%----------------------------------------------------------------------------
\chapter*{\eloszo}\addcontentsline{toc}{chapter}{\eloszo}
%----------------------------------------------------------------------------

Az előszó legtöbbször személyes hangú, eligazító jellegű írás, amely a mű megírásának okairól, születésének körülményeiről szól. Az előszó nem szerves része a főszövegnek, hanem annak kiegészítése.
Ugyancsak az előszóban fejtheti ki a szerző a mű megértéséhez szükséges szempontokat, a követett módszereket, utalhat a fontosabb előzményekre és szakirodalomra.
Az előszó ne legyen terjedelmes.


Jelen dokumentum egy diplomaterv-sablon, amely formai keretet ad a BME Gépészmérnöki Karán végző hallgatók által elkészítendő szakdolgozatnak és diplomatervnek. A sablon használata opcionális. Ez a sablon \LaTeX~alapú, a \emph{TeXLive} \TeX-implementációval és a PDF-\LaTeX~fordítóval működőképes.
A sablon forrása a Mechatronika Szakosztály GitHub tárhelyén\footnotemark{} elérhető. Amennyiben hibát találtál, vagy észrevételed, javaslatod lenne, kérlek ott jelezd.

\footnotetext{\url{https://github.com/MechatronikaSzakosztaly/bme-gpk-thesis-latex}}

\begin{center}
    $\thicksim \; \thicksim \; \thicksim$
\end{center}


\subsubsection*{Köszönetnyilvánítás}
\emph{Szeretnék köszönetet mondani témavezetőmnek, Dr. Budai Csabának és konzulensemnek Dudás Dávidnak a több féléves munkájukért. Szeretném megköszönni Dr. Bodnár Tibor, Bodnárné Dr. Gyarmathy Dórának és Dr. Gyarmathy Miklósnénak a támogatást és a lektorálást. Szeretnék köszönetet mondani Dr. Kiss Rita tanészékvezető asszonynak és teljes tanészéknek a támogatásért. Szeretném megköszönni az Orvostechnika Szakosztály tagjainak a projektben vállalt kisebb feladatokban nyújtott segítséget, illetve a lehetőséget, hogy ezzel a projekttel hozzájárulhattam a Szakosztály életéhez.} 


\vspace{0.5cm}

\begin{flushleft}
{Budapest, \today}
\end{flushleft}

\begin{flushright}
\emph{\authorName}
\end{flushright}

\vfill
