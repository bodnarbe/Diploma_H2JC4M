%----------------------------------------------------------------------------
\appendix
%----------------------------------------------------------------------------
\chapter*{\melleklet}\addcontentsline{toc}{chapter}{\melleklet}
%----------------------------------------------------------------------------
\setcounter{chapter}{\annexletter} % M betű
\setcounter{section}{0}
%\setcounter{equation}{0}
\numberwithin{equation}{section}
\numberwithin{figure}{section}
\numberwithin{lstlisting}{section}
%\numberwithin{tabular}{section}

%----------------------------------------------------------------------------
\section{Telemanipulátor mikrovezérlő programja}
\label{sec:melleklet_1}
%----------------------------------------------------------------------------
A TELEMANIP 2023 nevű mappa tartalmazza a mikrovezérlő programját és a
működéséhez szükséges könyvtárakat. A program STM32 CubeMX-ben nyitható meg.

\section{Dudás Dávid - Mogi Haptic Device}
\label{sec:melleklet_2}
%----------------------------------------------------------------------------
\url{https://github.com/dudasdavid/HapticDevice}

A linken elérhető a teljes eszköz elkészítéséhez és beüzemeléséhez szükséges információ.
Az általam felépített rendszerben felhasználtam, ennek a rendszernek a ROS
kommunikációhoz szükséges elemeket.

\section{ROS kiegészítő számítások}
\label{sec:melleklet_3}
%----------------------------------------------------------------------------
Ezek a programok szolgálnak a ROS rendszerben a direkt kinematika által a TCP pont
és annak orientációjának megállapítására.

\section{A teljes projekthez tartozó git repo}
\label{sec:melleklet_4}
%----------------------------------------------------------------------------
Itt minden szükséges kiegészítő dokumentumot meglehet találni, illetve a CAD modelljét is a telemanipulátornak.

\url{git@github.com:bodnarbe/Diploma_H2JC4M.git}

