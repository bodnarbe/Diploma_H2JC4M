%----------------------------------------------------------------------------
\chapter{\bevezetes}
%----------------------------------------------------------------------------

A diploma dolgozatomban bemutatásra kerülő telemanipulátorral kapcsolatos feladatok több szempontból is prototípus alapú feladatok. Nehezen lehet mérnöki életben próba, prototípus tesztelés nélkül elméleti számításokon alapuló rendszerekről következtetéseket levonni. A dolgozat tárgyát képezős telemanipulátor prototípusának tekinthető a szintén általam készített szakdolgozatomban bemutatott telemanipulátor. Annak az eszköznek az elkészülését követően is számtalan ötlet és kérdés merült fel bennem, hogy hogy tudtam volna jobban elkészíteni. A diploma feladatainak megfogalmazásának  idejére már egyértelművé vált számomra, hogy ugyan ezzel a témával szeretnék foglalkozni és szeretnék egy még jobb rendszert elkészíteni kiegészítve új funkciókkal.

A nehézségi erőből fakadó tehetetlenség kompenzációjának kérdése már foglalkoztatott a szakdolgozatom alatt is, de ott sajnos idő hiány miatt nem sikerült elmélyednem benne. Ezért kézen fekvő volt a diplomamunkám során ezzel kezdeni. Fontosnak tartottam azt a célt, hogy a sebészeti eszközök mozgatása legyen itt is a cél, ami felé orientálódnom kell, mivel egye nagyon nagy fokú precizitást és körültekintést igénylő terület és számtalan mérnöki kihívást tartogat. Ezt követően elkezdtem megtervezni az eszközt, de a prototípusok sora vezetett el mindig a következő lépcsőfokhoz. Igyekszem a dolgozatomban a lényegre törően megfogalmazni azokat az elvárásokat, amiket minden egyes rossz irányú tervezésnél vagy hibás következetesénél megfogalmaztam.

A következők dolgozatban a diplomamunkám során elvégzett munkámat mutatom be. Igyekszem logikailag az építési sorrendre való tekintettel felbontani a teljes feladatot és lépésről lépésre részletesen bemutatni minden kapcsolódó technikai részletet és döntést mit miért és mire használtam. A előbb említettek alapján röviden elemzem a szakdolgozatomban bemutatott telemanipulátort és azt azokat az elvárásokat amik alapján újra terveztem az egész eszközt. A valós tervezés megkezdése előtt a gravitációs hatásából származó terhelés kompenzációjának lehetőségeit mutatom be és extra elvárásként fogalmazom meg, hogy a tervezésnek figyelembe kell vennie ennek megvalósíthatóságát. Ezt követően a elvárások figyelembe vételével a geometriai kialakítás, az elektronikai feldolgozó rendszer és hozzátartozó mikrovezérlő program megtervezése és megvalósítását részletezném. Végül pedig a kész telemanipulátorral gyűjtött adatok feldolgozását és felhasználásának módját mutatnám be, amivel szimulációs majd valós robot kart is célom vezérelni.