%----------------------------------------------------------------------------
\chapter{\bevezetes}
%----------------------------------------------------------------------------

A diploma dolgozatomban részletesen bemutatom a telemanipulátor megtervezését. A feladat, amit ehhez el kellett végeznem egy sokrétű prototípusokon alapuló rekurzív tervezési folyamat. Nehezen lehet a mérnöki életben próba, prototípus tesztelés nélkül elméleti számításokon alapuló rendszerekről következtetéseket levonni. A dolgozat tárgyát képező telemanipulátor prototípusának tekinthető a szintén általam készített szakdolgozatomban bemutatott telemanipulátor. Annak az eszköznek az elkészülését követően is számtalan ötlet és kérdés merült fel bennem, hogy hogy tudtam volna jobban elkészíteni. A diplomamunkám feladatainak megfogalmazásának idejére már egyértelművé vált számomra, hogy ugyan ezzel a témával szeretnék foglalkozni. Célom lett ezzel egy még jobb rendszert elkészíteni kiegészítve új funkciókkal.

A nehézségi erőből fakadó tehetetlenség kompenzációjának kérdése már foglalkoztatott a szakdolgozatom alatt is, de ott sajnos időhiány miatt nem sikerült elmélyednem benne, ezért kézen fekvő volt a diplomamunkám során ezzel kezdeni. Fontosnak tartottam azt, hogy a sebészeti eszközök mozgatása legyen itt is a cél, ami felé orientálódnom kell, mivel ez a terület nagy fokú precizitást és körültekintést igényel és számtalan mérnöki kihívást tartogat. Ezt követően elkezdtem megtervezni az eszközt. A tervezés során minden nagyobb módosítás vezetett el mindig a következő lépcsőfokhoz, amiknél szinte kivétel nélkül prototípus tesztelést csináltam. Igyekszem a dolgozatomban lényegre törően megfogalmazni azokat az elvárásokat, amiket minden egyes helytelen irányú tervezésnél vagy hibás következetesénél megfogalmaztam.

A következő fejezetekben a diplomamunkám során elvégzett munkámat mutatom be. Igyekszem logikailag az építési sorrendre való tekintettel felbontani a teljes feladatot és lépésről-lépésre részletesen bemutatni minden kapcsolódó technikai részletet és döntést, mit miért és mire használtam. Az előbb említettek alapján röviden elemzem a szakdolgozatomban bemutatott telemanipulátort és azokat az elvárásokat amik alapján újra terveztem az egész eszközt. A valós tervezés megkezdése előtt a gravitációs hatásából származó terhelés kompenzációjának lehetőségeit mutatom be és extra elvárásként fogalmazom meg, hogy a tervezésnek figyelembe kell vennie ennek megvalósíthatóságát is. Ezt követően az elvárások figyelembe vételével a geometriai kialakítást, az jelfeldolgozó rendszert és a hozzátartozó mikrovezérlő programot megvalósítását részletezném. Végül pedig a kész telemanipulátorral gyűjtött adatok feldolgozását és felhasználásának módját mutatnám be, amivel szimulációs majd valós robotkart is célom vezérelni.